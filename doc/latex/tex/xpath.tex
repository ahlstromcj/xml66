%-------------------------------------------------------------------------------
% xpath
%-------------------------------------------------------------------------------
%
% \file        xpath
% \library     Documents
% \author      Chris Ahlstrom
% \date        2026-02-20
% \update      2026-02-24
% \version     $Revision$
% \license     $XPC_GPL_LICENSE$
%
%     This document provides LaTeX documentation for the xml66 library.
%
%-------------------------------------------------------------------------------


\section{XPath}
\label{sec:xpath}

   This section discusses the main points of \textsl{XPath}
   as it applies to the partial coverage of XML supported
   by the \textsl{Xml66} library.
   See \cite{xpath} for more complete information.
   An older specification is at \cite{xpath1999}.

   \textsl{XPath} stands for \textsl{XML Path Language}.
   It uses a non-XML syntax for a
   flexible way of addressing different parts of an XML document.
   It can also test addressed nodes within a document to determine
   if they match a pattern.
   \textsl{XPath} models an XML document as a tree of nodes.
   There are different types of nodes, including \textsl{element nodes},
   \textsl{attribute nodes}, and \textsl{text nodeso}.
   \textsl{XPath} defines a way to compute a string-value for each
   type of node.
   Some types of nodes also have names.
   XPath fully supports XML Namespaces.

   The syntax of a path is somewhat complex, so we will
   distill it into the information needed to use the
   \textsl{Xml66} library.

\subsection{Expr: expression}
\label{subsec:xpath_expr_expression}

   An \textsl{Expr} is the main syntactic construct of
   \textsl{Xpath}.
   It is evaluated, yielding one of these four types:

   \begin{itemize}
      \item \textbf{Node-set}.
         An unordered collection of nodes without duplicates.
      \item \textbf{Boolean}.
         A true or false value.
      \item \textbf{Number}.
         A floating-point number.
      \item \textbf{String}.
         A sequence of UCS characters.
   \end{itemize}

\subsection{Context}
\label{subsec:xpath_context}

   Expression evaluation occurs in a \textsl{Context}.
   The context consists of:

   \begin{itemize}
      \item \textbf{Node}.
         This is the "context"node.
      \item \textbf{Context position and size}.
         This is a pair of non-zero positive integers.
      \item \textbf{Set of variable bindings}.
         These are a mapping from variable names to
         variable values (each value is an object; see the
         previous section).
      \item \textbf{Function library}.
         These are a mapping from function names to
         functions.
      \item \textbf{Set of namespace declarations}.
         These consist of a mapping from prefixes to namespace URIs.
   \end{itemize}

\subsection{Location Path}
\label{subsec:xpath_location_path}

   The \textsl{Location Path} is a special case of an
   \textsl{Expr}, and is the most important construct.

\subsection{Future Work}
\label{subsec:xpath_introduction_future}

   \begin{itemize}
      \item Hammer on this documentation.
   \end{itemize}

%-------------------------------------------------------------------------------
% vim: ts=3 sw=3 et ft=tex
%-------------------------------------------------------------------------------
